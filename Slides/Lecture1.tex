\documentclass{beamer}
\usetheme{metropolis}           % Use metropolis theme
\title{Financial Crisis and Macroeconomic Policy}
\date{\today}
\author{Pablo Cuba Borda}
\institute{Universidad Católica Boliviana San Pablo // Semestre I, 2025}
\begin{document}
  \maketitle

\section{Logística}
\begin{frame}{Logística}

	\begin{itemize}
		\item Introduccíon al curso
		\item Revisión del Syllabus
		\begin{itemize}
			\item 	\textbf{Horario:} Martes y Jueves, 19:45 - 21:00 \\
			\item \textbf{Evaluación: }
				\begin{itemize}
				\item 	\textbf{Ensayo cortos:} 1.000 palabras (para esta parte del curso), \textbf{6 de marzo}
				\item 	\textbf{Trabajo final:} 10-15 páginas, \textbf{al finalizar el curso, fecha por definir}. 
				\item 	\textbf{Propuestas trabajo final:} 10 minutos, 5 slides, textbf{11 y 13 de marzo}. \textcolor{red}{Opcional! confirmar interes hasta el 6 de marzo}.
			\end{itemize}
			\item Clases: slides en ingles, discusión en español	
			\item Lecturas: disponibles en  https://github.com/pcuba/POLECONUCB.
			\item Contacto: pcubabor@umd.edu 
		\end{itemize}
		\item ¿Preguntas?

	\end{itemize}

	

\end{frame}

  \section{Introduction}
  \begin{frame}{Motivation}
	  \begin{itemize}
	  \item Crises are dramatic events 
	  \item Share common elements across time and space
	  \begin{itemize}
	  \item Loss of confidence on ability to repay financial obligations (governments, banks, private agents)
	  \item Exacerbated by fiscal and monetary policies
	  \item Expectations play a crucial role
	  \end{itemize}
	  \item Depend on economic context: trade and financial integration, exchange rate regimes, policies and institutions. 
	  \end{itemize}
  \end{frame}

\begin{frame}{Some examples}
Let's look at some examples
\centering
\begin{tabular}{lc} \hline
Chile & 1926, 1980, \\
Mexico & 1929, 1981 \\
Philippines & 1981\\
Indonesia & 1997\\ 
Greece & 2007 \\
Ukraine & 2008 \\
Iceland & 2008 \\

Italy & 1866, 1921, 2008 \\
Argentina & 1890, 1931/1934, 1980/1985, 2001 \\ \hline 
\end{tabular}
\end{frame}

  \begin{frame}{Key elements to consider}
	\begin{itemize}
	\item Capital flows
	\item Domestic credit 
	\item Exchange rate and interest rate policies
	\item Balance sheet deterioration (banks, households, government)
	\item Financial frictions
	\item Interactions
	\end{itemize}
  \end{frame}
  
  \begin{frame}{Capital Flows}
  \centering
	\includegraphics[width=10cm]{../Data/netflows.png}
  \end{frame}
  
    \section{Anatomy of Crises}
  \begin{frame}{Credit Booms}

\begin{enumerate}
\item 	Credit expansions are associated with boom-bust dynamics in:
  \begin{itemize}
  \item Economic activity
  \item Equity and housing prices
  \item Real exchange rate (appreciation followed by a crash)
  \item External deficits
  \end{itemize}
\item  Credit booms in AE and EE share similar characteristics. 
\item  Are synchronized internationally around major (global) events. 
\item  Not all credit booms end up in crisis, but are often followed by banking crisis, currency crisis and sudden stops.  
\end{enumerate}
  
  \end{frame}

\begin{frame}{Measuring Credit Booms}
Follow methodology of Mendoza and Terrones (2012):
\begin{itemize}
\item Define country-i is in a credit boom if:
\begin{equation*}
l_{i,t} \geq \phi \sigma(l_i)
\end{equation*}
$\phi$: threshold factor (e.g. 1, 1.5, 2,...) \\
$l_{i,t}$: deviation of $\log(credit/per-capita)$ relative to trend (e.g. HP filter)\\
$\sigma(l_i):$ standard deviation of $l_{i,t}$
\item Set $\phi=1.65$ because $Pr(l_{i,t}/\sigma(l_i)\geq 1.65) = 0.05$
\item Consider $5\%$ of tail events of credit-growth distribution
\end{itemize}
\end{frame}

\begin{frame}{Event study}
\centering
\includegraphics[width=10cm]{../Data/credit.png}
\end{frame}

\begin{frame}{Event study}
\centering
\includegraphics[width=10cm]{../Data/output.png}
\end{frame}

\begin{frame}{Event study}
\centering
\includegraphics[width=10cm]{../Data/consumption.png}
\end{frame}

\begin{frame}{Event study}
\centering
\includegraphics[width=10cm]{../Data/investment.png}
\end{frame}


\begin{frame}{Event study}
\centering
\includegraphics[width=10cm]{../Data/government.png}
\end{frame}


\begin{frame}{Event study}
\centering
\includegraphics[width=10cm]{../Data/outputnt.png}
\end{frame}

\begin{frame}{Event study}
\centering
\includegraphics[width=10cm]{../Data/rer.png}
\end{frame}

\begin{frame}{Event study}
\centering
\includegraphics[width=8.25cm]{../Data/external.png}
\begin{itemize}
\vspace*{-0.35cm}
\item Question: What happens with real money balances?
\end{itemize}
\end{frame}

\begin{frame}{Event study}
\centering
\includegraphics[width=9cm]{../Data/prices.png}
\begin{itemize}
\vspace*{-0.25cm}
\item Question: What about aggregate inflation? Why?
\end{itemize}
\end{frame}


\begin{frame}{Summary}

\begin{itemize}
\item Credit booms across AE and EE associated with boom-bust dynamics.
\item Output, expenditures, stock prices, housing prices, and the real exchange rate move above trend during the build-up phase.
\item Drop below trend in the bust-phase, the current account falls first and then rises.
\item No major changes in inflation.
\item EMs display: larger fluctuations,  procyclical fiscal policy, sudden stop features (sharp swings in CAY, P-NT, Y-NT).
\end{itemize}

Next: Association of credit booms and financial crisis? Potential drivers of credit booms? 
\end{frame}


\begin{frame}{Credit booms and financial crises}


\begin{table}
%\caption{Joint frequency distribution}
\begin{tabular}{cccc}
& Banking Crisis & Currency Crisis & Sudden Stops \\ \hline
All countries & 0.44 & 0.54 & 0.24 \\
AE countries & 0.36 & 0.44 & 0.14 \\
EE countries & 0.51 & 0.63 & 0.34 \\ \hline
\end{tabular}
\end{table}
\begin{itemize}
\item Not all credit booms end in financial crises but 1 out fo 4 do
\item Financial crises tend to be at its highest after credit booms peak (both in AE and EM)
\item Consistent with Schularick and Taylor (2012) credit growth is a predictor of banking crisis 
\item macro fluctuations in the countries that experienced crisis
are larger and display more abrupt declines than those of the non-crisis countries
\end{itemize}
\end{frame}


\begin{frame}{What drives credit booms?}
\centering
\includegraphics[width=11cm]{../Data/creditboomdrivers.png}
\begin{itemize}
\vspace*{-0.25cm}
\item Surges in capital inflows tend to precede credit booms
in both AE and EE
\item TFP growth and financial reform lead to credit booms in AE, but less so in EE
\end{itemize}
\end{frame}

\begin{frame}{More evidence}
\begin{itemize}
\item Take away: credit and capital flows tightly linked
\end{itemize}
\centering
\includegraphics[width=9cm]{../Data/netflowsandcredit.png}
\end{frame}

\begin{frame}
\frametitle{Additional results}
\begin{itemize}
\item $\sim 70\%$
of credit booms occur in countries with managed or fixed exchange rate regimes (AE and EE)
\item Once credit grows above $1\sigma$ above trend, probability of a credit boom is $17\%$. AE $23 \%$ and EM $13\%$
\end{itemize}
\end{frame}





\end{document}